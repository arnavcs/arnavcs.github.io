\section{Projects}

\begin{resumeItem}
\projectHead{\href{https://github.com/arnavcs/4d-raymarching-pathtracer}{4D Raytracing Pathtracer}}{C++}{Jul - Aug 2025}
\begin{resumeList}
    \item Implemented SDF raymarching in 4D space with CPU pathtracing to contruct GIF renders of the scene.
    \item Encoded the render such that every frame in the GIF, the camera sends out rays into a 3D affine subspace of the 4D space, which are scattered into 4D space upon colliding at a surface.
    \item Designed and wrote a powerful material system with microfacet BRDFs, glass, volumetric fog, and more with the ability to assign more than one material to a geometry.
\end{resumeList}
\end{resumeItem}


\ifthenelse{ \equal{\focus}{general} }{
\begin{resumeItem}
\projectHead{Rigid-body Particle Simulation}{C++}{Jul 2025}
\begin{resumeList}
    \item Simulated gravity, spherical constraints, and collision physics for rigid body particles.
\end{resumeList}
\end{resumeItem}
}{}

\begin{resumeItem}
\projectHead{Software Rasterizer}{C++}{May 2025}
\begin{resumeList}
    \item Recreated the whole rasterization process on the CPU, including perspective transformation, triangle rasterization, depth buffering, and perspective correct interpolation.
\end{resumeList}
\end{resumeItem}

\begin{resumeItem}
\projectHead{Software Raytracer}{C++}{May - Jun 2025}
\begin{resumeList}
    \item Implemented the Möller–Trumbore algorithm for fast ray-triangle intersection
    \item Programmed Lambertian diffuse, specular refraction, and specular reflection behaviour, and support for spherical environment mapping
\end{resumeList}
\end{resumeItem}

% \begin{resumeItem}
% \projectHead{\href{https://github.com/arnavcs/software-rasterizer}{Software Rasterizer}}{Elm}{May 2025}
% \begin{resumeList}
%     \item Created rasterizing essentials such as drawing shaded triangles pixel by pixel
%     \item Implemented a projection algorithm to view 3D objects from a camera.
% \end{resumeList}
% \end{resumeItem}

\begin{resumeItem}
\projectHead{\href{https://arnavcs.itch.io/renovating-the-labyrinth}{Renovating the Labyrinth (Game)}}{JavaScript}{Oct 2024}
\begin{resumeList}
    \item Solo submission made in 72 hours with vanilla JavaScript on an HTML canvas for the UW Game Dev Club's fall 2024 game jam; voted winner of the technical achievement award.
    \item Built a real time optimized ray caster with ordered Bayer matrix dithering.
    \item Programmed 2D rigid body collision behaviour and a randomized Prim's algorithm for map generation.
\end{resumeList}
\end{resumeItem}

\ifthenelse{ \equal{\version}{cv} }{
\ifthenelse{ \equal{\focus}{general} }{
\begin{resumeItem}
\projectHead{\href{https://arnavcs.github.io/software-raycaster}{Software Raycaster}}{JavaScript}{Oct 2024}
\begin{resumeList}
    \item Built a raycaster that renders a specified scene to a canvas element.
    \item Enabled options for Bayer matrix dithering, scene customization, and different casting algorithms.
\end{resumeList}
\end{resumeItem}
}{}

\ifthenelse{ \equal{\focus}{general} }{
\begin{resumeItem}
\projectHead{\href{https://kdxiao.itch.io/bloom}{Bloom (Game)}}{Godot, GDScript}{Jun 2024}
\begin{resumeList}
    \item Team submission made in 72 hours with Godot for the UW Game Dev Club's spring 2024 game jam.
    \item Implemented colour mixing, screen wrapping, movement, and flower spawning mechanics.
\end{resumeList}
\end{resumeItem}
}{}
}{}

\ifthenelse{ \equal{\focus}{general} }{
\begin{resumeItem}
\projectHead{\href{https://github.com/youssefsoli/IPFE}{InterPlanetary File Explorer (IPFE)}}{Go, Python, Scikit-learn, Estuary, Co:here, Three.js}{Jan 2023}
\begin{resumeList}
    \item Created vector embeddings for files with their headers using Co:here's NLP embeddings to facilitate classification of files.
    \item Performed principal component analysis of the vector embeddings to reduce the dimensionality from 4096 to 3 to be plotted and displayed interactively in 3D space using Three.js.
\end{resumeList}
\end{resumeItem}
}{}

\ifthenelse{ \equal{\focus}{general} }{
\begin{resumeItem}
\projectHead{\href{https://github.com/arnavcs/OSIC-IPF}{Prognosing Idiopathic Pulmonary Fibrosis (IPF)}}{Python, Tensorflow2, Pandas, Scikit-learn}{Dec 2020 - Jun 2021}
\begin{resumeList}
    \item Implemented an auto-encoder, linear regression, dense neural network, and bayesian model in order to accurately predict future lung capacity and give a confidence value using initial lung capacity data, age, sex, smoking status, and more.
    \item Obtained a Laplace Log Likelihood score of $-6.9$ (much better than the baseline score $-8.1$) with $\sigma \approx 200$mL.
\end{resumeList}
\end{resumeItem}
}{}

% \begin{resumeItem}
% \projectHead{\href{https://github.com/arnavcs/quack-stack}{Custom Language Interpreter}}{Haskell}{May 2023}
% \begin{resumeList}
%     \item Created an expression evaluator for a self-made stack-based language.
%     \item Implemented zipper traversal to determine which part of the stack has already been processed.
% \end{resumeList}
% \end{resumeItem}
 
% \begin{resumeItem}
% \projectHead{\href{https://github.com/arnavcs/cli-chess}{CLI Chess}}{Haskell}{Aug 2022}
% \begin{resumeList}
%     \item Created a local mulitplayer chess clone in Haskell that can be played in the CLI with the Haskell REPL.
%     \item Developed safe and scalable functions to determine the state of the board and indentify check and checkmate.
%     \item Employed lists as applicative functors and monads to write concise, readable code for determining all valid next moves.
% \end{resumeList}
% \end{resumeItem}

% \begin{resumeItem}
% \projectHead{\href{https://github.com/arnavcs/flashcards}{CLI Flashcards}}{Bash}{Apr 2022 - Apr 2022}
% \begin{resumeList}
%     \item Built a command line tool to add terms to and practice with a flashcard set that the user can create.
%     \item Implemented several flag options, including a help option, version option, and script update option.
% \end{resumeList}
% \end{resumeItem}

% \begin{resumeItem}
% \projectHead{\href{https://github.com/arnavcs/ASCII-art}{Braille ASCII Art Generator}}{Python, OpenCV}{Apr 2021 - Jul 2021}
% \begin{resumeList}
%     \item Used thresholding in OpenCV to determine which pixels of the output art should be shaded.
%     \item Transformed the image array into braille characters using 3x2 pixel filters which returned a binary number corresponding to a braille character.
% \end{resumeList}
% \end{resumeItem}

\ifthenelse{ \equal{\version}{cv} }{
\ifthenelse{ \equal{\focus}{general} }{
\begin{resumeItem}
\projectHead{\href{https://github.com/arnavcs/black-jack}{Blackjack}}{C++}{Apr 2021 - Jul 2021}
\begin{resumeList}
    \item Used object oriented programming to implement a local multiplayer blackjack game with a computer playing as the house.
    \item Probabilistically optimized the holding value that the computer house uses.
\end{resumeList}
\end{resumeItem}
}{}
}{}

% \begin{resumeItem}
% \projectHead{\href{https://github.com/arnavcs/game-of-life}{Conway's Game of Life}}{Javascript}{Apr 2020 - Mar 2020}
% \begin{resumeList}
%     \item Created a javascript web version of Conway's Game of Life where a user-chosen number of generations can be simulated.
%     \item Enabled the user to choose an initial arrangement of live and dead cells as well as the size of the initial grid.
% \end{resumeList}
% \end{resumeItem}

